The Minesweepers Competition calls for autonomous, ROS-based (Robot Operating System), multi-robot systems which robustly detect and map surface and buried landmines in a controlled but challenging environment.

CVRA builds a rover on a rocker-bogie locomotion system to tackle rough terrain.
Precise global positioning is achieved through a custom localization system which measured the Round Trip Time of a radio signal between the rover and four fixed beacons placed along the edge of the playing field.
Navigation and path planning will be handled by a human operator in a first version, but ROS in combination with the rover's depth sensor lay the foundation for doing this autonomously.  
The RGBD (Red Green Blue Depth) sensor is used to detect surface mines visually.
An Intel NUC computer provides the computing power for this task.  
Buried mines are found with an array of custom built pulse induction metal detectors.
The whole system is based around an on-board computer running ROS, to which a number of independent hardware-interacting modules connect via CAN-bus.

Challenges are the integration of all necessary subsystems to build a functioning robot in a limited amount of time, and to get this robot to detect landmines at the speed required by the competition.
