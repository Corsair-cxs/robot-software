The code of this project can be separated into three different
categories: libraries, portable code and platform specific code.
Libraries are Open Source code that are re-used from other projects.
Portable code is generic code that does not contain anything specific to
this processor or board, but is only useful in this project. It can be
moved into a library if another project requires it later. Portable code
is usually well covered by unit tests (see the \texttt{tests} folder),
where non portable code must usually be tested ``by hand''.

\section*{Libraries}\label{libraries}

Most of those libraries have READMEs that describe them in much more
detail.

\begin{itemize}
\tightlist
\item
  \texttt{chibios-syscalls} provides a few extensions for ChibiOS (our
  RTOS) for malloc and printf mostly.
\item
  \texttt{crc} implements some common checksums algorithms.
\item
  \texttt{decadriver} is provided by Decawave to interact with their
  product.
\item
  \texttt{eigen} is a C++ matrix manipulation library used for the EKF.
\item
  \texttt{parameter} provides a unified interface for the application to
  declare parameters.
\item
  \texttt{cmp}, \texttt{cmp\_mem\_access} and
  \texttt{parameter\_flash\_storage} are used to store the parameters in
  non volatile memory.
\item
  \texttt{msgbus} implements a software message bus
  (publisher/subscriber pattern), similar in concept to D-bus or ROS.
\item
  \texttt{libuavcan} is the official UAVCAN stack.
\item
  \texttt{test-runner} is used only to provide an environment during
  unit testing.
\end{itemize}

\section*{Portable code}\label{portable-code}

\begin{itemize}
\tightlist
\item
  \texttt{MadgwickAHRS.\{c,h\}} contains the implementation of the
  Madgwick filter.
\item
  \texttt{lru\_cache.\{c,h\}} contains a generic implementation of a
  Least Recently Used (LRU) cache.
\item
  \texttt{mpu9250.\{c,h\}} contains a portable implementation of a
  driver for the Invensense MPU9250 IMU.
\item
  \texttt{state\_estimation.\{cpp,h\}} contains the code for the state
  estimator (uses ekf.hpp).
\item
  \texttt{ekf.hpp} contains a generic implementation of an EKF using C++
  templates.
\item
  \texttt{uwb\_protocol.\{c,h\}} contains a portable implementation of
  the UWB protocol.
\end{itemize}

\section*{Non portable code}\label{non-portable-code}

\begin{itemize}
\tightlist
\item
  \texttt{ahrs\_thread.\{c,h\}} contains the thread that runs the
  Madgwick filter.
\item
  \texttt{anchor\_position\_cache.\{c,h\}} contains the thread that
  stores the position of the anchors and provides them to the other
  threads.
\item
  \texttt{imu\_thread.\{c,h\}} contains the thread that runs the IMU
  driver.
\item
  \texttt{ranging\_thread.\{c,h\}} contains the thread responsible for
  doing range measurements and other radio operations.
\item
  \texttt{state\_estimation\_thread.\{cpp,h\}} contains the thread that
  runs the state estimation.
\item
  \texttt{usbconf.\{c,h\}} contains the USB setup code.
\item
  \texttt{main.\{c,h\}} contains the entrypoint of the application.
\item
  \texttt{cmd.\{c,h\}} contains the command line interface shown on USB.
\item
  \texttt{decawave\_interface.\{c,h\}} contains the low level bindings
  between Decawave's library and our code.
\item
  \texttt{board.\{c,h\}} contains GPIO contains GPIO configuration.
\item
  \texttt{bootloader\_config.\{c,h\}} contains code for interacting with
  the CAN bootloader used on some boards.
\item
  \texttt{exti.\{c,h\}} contains the external interrupt configuration
  and handlers.
\item
  The \texttt{uavcan} folder contains code related to moving messages
  from the application to the CAN bus and vice versa.
\end{itemize}
